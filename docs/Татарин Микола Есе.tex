% Created 2023-05-08 Mon 14:04
% Intended LaTeX compiler: xelatex
\documentclass{scrartcl}
\usepackage{fontspec}
\usepackage[ukrainian]{babel}
\usepackage{biblatex}
\addbibresource{~/Org/.bib}
\usepackage{tikz}
\usepackage{graphicx}
\usepackage{longtable}
\usepackage{wrapfig}
\usepackage{rotating}
\usepackage[normalem]{ulem}
\usepackage{amsmath}
\usepackage{amssymb}
\usepackage{unicode-math}
\usepackage{capt-of}
\usepackage{hyperref}
% style: default
% features: (font-main font-sans font-mono)
\setmainfont{Alegreya}
\setsansfont{Alegreya Sans}
\setmonofont[Scale=MatchLowercase]{Iosevka}
% end features
\author{Татарин Микола, ІТ-92}
\date{\today}
\title{Есе по темі дипломної роботи для першої зустрічі з керівником}

\providecolor{url}{HTML}{0077bb}
\providecolor{link}{HTML}{882255}
\providecolor{cite}{HTML}{999933}
\hypersetup{
  pdfauthor={Татарин Микола, ІТ-92},
  pdftitle={Есе по темі дипломної роботи для першої зустрічі з керівником},
  pdfkeywords={},
  pdfsubject={},
  pdfcreator={Emacs 29.0.90 (Org mode 9.6.1)},
  pdflang={Ukrainian},
  breaklinks=true,
  colorlinks=true,
  linkcolor=link,
  urlcolor=url,
  citecolor=cite
}
\urlstyle{same}
% hide links styles in toc
\NewCommandCopy{\oldtoc}{\tableofcontents}
\renewcommand{\tableofcontents}{\begingroup\hypersetup{hidelinks}\oldtoc\endgroup}
\begin{document}

\maketitle

\section{Тема}
\label{sec:orgc05631d}
Застосунок для автоматичної категоризації художніх творів.

\section{Проблема, що розглядається}
\label{sec:orgd69c946}
На сьогоднішний день у світі існує величезна кількість текстової інформації (книг, статей, блогів, і т.д.). Тому знайти те, що тобі потрібно, іноді досить важко. Те, наскільки легко чи важко можна знайти той чи інший шматок інформації, залежить від декількох факторів. Один із таких факторів - це метадані, як от: заголовок, опис, ключові слова. Останні особливо корисні, бо дозволяють користувачу пошукової системи звузити пошук до тої теми, яка йому потрібна. В ідеалі ключові слова повинен надати автор твору, але іноді цього не відбувається. Або ж автор це зробив, але метадані були якимось чином втрачені. Тоді для їх відновлення хтось повинен буде перечитати цей твір, і знову скласти список ключових слів. Деякі платформи (наприклад, goodreads) дозволяють читачам колективно редагувати теги, проте було б непогано, якби це можна було робити автоматично.

\section{Мета проекту}
\label{sec:org6bcef2a}
Розробити застосунок для автоматичної категоризації текстів (я планую обмежитись художньою літературою), і співставлення цих текстів з набором ключових слів (наприклад: {}<<історія>>{}, {}<<драма>>{}). Використати для цього машинне навчання.

\section{Що планується зробити}
\label{sec:orgc6da442}
\begin{enumerate}
\item Розібрати вже існуючі моделі для вирішення цієї задачі і обрати оптимальну.
\item Підібрати існуючий датасет категоризованих за тегами текстів, або ж створити свій, використовуючи літературу з публічного домену.
\item Навчити на цьому датасеті обрану модель.
\item Створити простий для використання інтерфейс, що дозволить використовувати цю модель в довільних застосунках.
\item Створити веб або десктоп застосунок для автоматичної категоризації художніх творів.
\end{enumerate}

\section{Де це можна використати}
\label{sec:org90f31e1}
Це можна використати для автоматичної категоризації і каталогування творів у онлайн каталогах (наприклад, goodreads) та бібліотеках.
\end{document}